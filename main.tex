%#!platex
\documentclass[platex,dvipdfmx]{rbproceedings}
% English option
%\documentclass[platex,dvipdfmx,english]{rbproceedings}
%#!uplatex
%\documentclass[uplatex,dvipdfmx]{rbproceedings}
%#!lualatex
%\documentclass[lualatex]{rbproceedings}

% パッケージ
\usepackage{graphicx,xcolor}  % グラフィックス関連
\usepackage{url}
\usepackage{hyperref}
\hypersetup{
    colorlinks=true,
    citecolor=blue,
    linkcolor=blue,
    urlcolor=blue,
    pdfborder={0 0 0},
}
%\usepackage{jlreq-deluxe}  % 多書体化(otf パッケージは使用しない)、Ubuntu 22.04 以降
\usepackage[verb]{bxghost}  % \verb 前後に適切な和欧文間スペース
\usepackage{pxrubrica}  % ルビ

% 参考文献のフォントサイズを指定
%\renewcommand{\bibfont}{\normalsize}  % 標準サイズ
%\renewcommand{\bibfont}{\footnotesize}  % より小さく

% \emph をゴシックかつ太字に(比較的新しい LaTeX が必要)
\DeclareEmphSequence{\gtfamily\sffamily\bfseries}

% 著者用マクロ
\newcommand{\pkg}[1]{\textsf{#1}}
\newcommand{\code}[1]{\texttt{#1}}
\newcommand{\comment}[1]{\textcolor{red}{#1}}

% タイトル
\title{\pkg{RBProceedings}文書クラス サンプル文書}

\author{%
佐藤 **${}^{1}$,鈴木 **${}^{1}$,高橋 **${}^{2}$,田中 **${}^{3}$\\
伊藤 **${}^{1,3}$,渡辺 **${}^{1,4}$\\ \\
${}^{1}$○○大学,${}^{2}$△△大学,${}^{3}$××株式会社,${}^{4}$□□研究所
}

\begin{abstract}
概要の例文.概要の例文.概要の例文.概要の例文.概要の例文.概要の例文.概要の例文.概要の例文.概要の例文.概要の例文.概要の例文.概要の例文.概要の例文.概要の例文.概要の例文.概要の例文.概要の例文.概要の例文.概要の例文.概要の例文.
\end{abstract}

% 本文
\begin{document}
\maketitle

\section{はじめに}
\pkg{RBProceedings}文書クラスはW3Cにより策定されている『日本語組版の要件』\cite{JLREQ}に準拠することを目指す\pkg{jlreq}クラスをベースにしている.
ただし,本文書クラスでは紙面スペースの都合上,多くの余白値をかなり詰めるように設定しており,例えば行間は\ruby{外国人参政権}{がい|こく|じん|さん|せい|けん}のようにルビを振れる最小限の余白に設定してある.

論文では,単純なテキストのみならず,しばしば数式
\begin{equation}
P(B\mid A) = \frac{P(A\mid B)P(B)}{P(A)}
\end{equation}
や箇条書き
\begin{itemize}
\item 第1の項目
\item 第2の項目
\end{itemize}
といった構造も用いられるが,これらもよく知られた文書クラス(例えば\pkg{jsarticle}等)と同様のシンタックスで利用できる.

\section{図表の挿入}
図表についても通常の \LaTeX と同じ方法を用いることができる.

\subsection{図について}
図の挿入は,通常\pkg{graphicx}パッケージによって行う(図\ref{fig:sample}).
クラスオプションにワークフロー(\code{dvipdfmx}等)を指定していれば,
各パッケージを読み込む際に何度も同じオプションを指定する必要はない.

\begin{figure}[t]
\centering
\includegraphics[width=3cm]{example-image-a}
\caption{図の例}
\label{fig:sample}
\end{figure}

\subsection{表について}
表の挿入は,\verb|\begin{table}...\end{table}|環境を使う(表\ref{tab:sample}).

\begin{table}[t]
\centering
\caption{表の例}
\label{tab:sample}
\begin{tabular}{llcc}
\hline
日本語 & Japanese & ほげほげ & ふげふげ \\
英語 & English & hogehoge & fugefuge \\
\hline
\end{tabular}
\end{table}

\section{参考文献}
参考文献の参照例.
\begin{itemize}
\item 論文誌の参照例 \cite{Article_01}
\item 本の参照例 \cite{Book_02}
\item 国際会議の参照例 \cite{Inproc_03}
\item 技術報告の参照例 \cite{Techrep_05}
\item Webページの参照例 \cite{Web_06}
\end{itemize}

\section{Writing in English}
This paragraph shows an English sample.
There is no problem with writing your manuscript in English.
If you write in LaTeX, please use the distributed document class with the \code{english} option:
\begin{quote}
\verb|\documentclass[|\\
\verb|  platex,dvipdfmx,english]{rbproceedings}|
\end{quote}

% 参考文献
\bibliographystyle{junsrt}
\bibliography{myrefs}

\end{document}
