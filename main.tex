%#!platex
\documentclass[platex,dvipdfmx]{rbproceedings}
% English option
%\documentclass[platex,dvipdfmx,english]{rbproceedings}
%#!uplatex
%\documentclass[uplatex,dvipdfmx]{rbproceedings}
%#!lualatex
%\documentclass[lualatex]{rbproceedings}

% パッケージ
\usepackage{graphicx,xcolor}  % グラフィックス関連
\usepackage{here}
\usepackage{url}
\usepackage{hyperref}
\hypersetup{
    colorlinks=true,
    citecolor=blue,
    linkcolor=blue,
    urlcolor=blue,
    pdfborder={0 0 0},
}
%\usepackage{jlreq-deluxe}  % 多書体化(otf パッケージは使用しない)、Ubuntu 22.04 以降
\usepackage[verb]{bxghost}  % \verb 前後に適切な和欧文間スペース
\usepackage{pxrubrica}  % ルビ

% 参考文献のフォントサイズを指定
%\renewcommand{\bibfont}{\normalsize}  % 標準サイズ
%\renewcommand{\bibfont}{\footnotesize}  % より小さく

% \emph をゴシックかつ太字に(比較的新しい LaTeX が必要)
\DeclareEmphSequence{\gtfamily\sffamily\bfseries}

% 著者用マクロ
\newcommand{\pkg}[1]{\textsf{#1}}
\newcommand{\code}[1]{\texttt{#1}}
\newcommand{\comment}[1]{\textcolor{red}{#1}}

% タイトル
\title{\pkg{RBProceedings}文書クラス サンプル文書}

\author{%
中村 勇太${}^{1}$,吉田 侑樹${}^{1}$,青木 修平${}^{1}$,村上 オト${}^{1}$\\
${}^{1}$Abudori Lab.
}

%\begin{abstract}
%概要の例文.概要の例文.概要の例文.概要の例文.概要の例文.概要の例文.概要の例文.概要の例文.概要の例文.概要の例文.概要の例文.概要の例文.概要の例文.概要の例文.概要の例文.概要の例文.概要の例文.概要の例文.概要の例文.概要の例文.
%\end{abstract}

% 本文
\begin{document}
\maketitle

\section{はじめに}
つくばチャレンジは2024年のロボット大賞に選出された。
つくばチャレンジは18年間、課題や取り組みを変えながら日本のロボット技術の進歩に大きく貢献してきた。
さまざまな研究者に大体的に実験できる環境を提供しつづけロボットの技術を成長させてきた。
完走しているチームの技術力は非常に高い。
初めて参加するチームや個人で参加するチームとは技術的な差が大きい。
初参加から完走するまでの道のりは遠い。
その要因は以下の点であると感じた。
完走しているチームの多くは高価な3DLiDARを搭載している。
ROS 2を使用した3Dナビゲーションのノウハウがウェブ上にあまり公開されていない。
つくばチャレンジレポートは昨年のレポートの差分が書かれていることが多い。
つくばチャレンジレポートは特定の課題の解法が書かれていることが多い。
つくばチャレンジレポートは過去のレポートがオープンになっていないため、引用先が見られない。
以上の点から、初学者や新規参入者に対して敷居が高いように感じた。
同時に、完走するために必要なことは以下のように感じた。
屋外の自律走行には初学者ほど3DLiDARがあった方がよい。
複雑なシステムを運用するには、質の良いハードウェアが必要である。
OSSを駆使すればある程度まではロボットを動作させることができる。
OSSを利用するだけでは完走には至らない。
AbudoriLab.チームでは、本走行では確認走行区間をクリアできなかった。
他のロボットが通路を塞いだため新ルートを作れなかった。
決められたルートに何も障害物がない状態であれば、もっと走行することができた。
決められたルートを走行することだけであれば、自己位置を喪失することなく、障害物も発見し走行することができた。
ゼロからつくばチャレンジに参加する人に必要なことがわかるレポートにしたい。
AbudoriLab.が用意したロボットやソフトウェア構成を紹介する。
\begin{figure*}[bhtp]
    \centering
    \begin{minipage}[b]{0.45\hsize}
       \centering
       \includegraphics[width=\hsize]{fig/component_parts.png}
       \caption{左の図}
       \label{fig:hidari}
    \end{minipage}
  %
    \begin{minipage}[b]{0.25\hsize}
       \centering
       \includegraphics[width=\hsize]{fig/exploded_view.png}
       \caption{右の図}
       \label{fig:migi}
    \end{minipage}
  \end{figure*}
\section{図表の挿入}
図表についても通常の \LaTeX と同じ方法を用いることができる.

\subsection{図について}
図の挿入は,通常\pkg{graphicx}パッケージによって行う(図\ref{fig:sample}).
クラスオプションにワークフロー(\code{dvipdfmx}等)を指定していれば,
各パッケージを読み込む際に何度も同じオプションを指定する必要はない.

\begin{figure}[t]
\centering
\includegraphics[width=3cm]{example-image-a}
\caption{図の例}
\label{fig:sample}
\end{figure}
\begin{figure*}[htbp]
    \centering
    \begin{minipage}[b]{0.45\hsize}
       \centering
       \includegraphics[height=5cm]{fig/tsukuba_map.jpeg}
       \caption{作成した3D地図}
       \label{fig:tsukuba_map}
    \end{minipage}
  %
    \begin{minipage}[b]{0.45\hsize}
       \centering
       \includegraphics[height=5cm]{fig/trajectry.jpeg}
       \caption{自己位置推定した軌跡}
       \label{fig:trajectry}
    \end{minipage}
  \end{figure*}
\section{地図作成}
現代の多くの自律走行ロボットは、行動範囲分の環境地図を必要とする。
ロボットが環境地図を持つことにより、既知の環境内で自由に行動することを目指す。

物流倉庫やオフィスビルなどの環境では2Dの地図を使用しても充分に効果を発揮する。
しかし、つくばチャレンジのようなひらけた場所であったり、交通の往来があったりする屋外環境であると2D地図の使い方によっては、性能が不十分になることがある。
このつくばチャレンジ2024では、回転式3DLiDARを利用し3D地図を作成した。
3D地図は複雑な環境や、本走行のように人だかりでロボット周辺に障害物がたくさんあってもロバストに自己位置推定を続けることができた。
回転式3DLiDARは非常に高価であるが、つくばチャレンジでは、無料貸与があるため、参加者は利用できるチャンスがある。
つくばチャレンジ2024では、株式会社アルゴさまより無料で貸与いただいた。
ここにお礼のことば(謝辞行きかどうかは要検討)

\subsection{GLIM}
地図作成には、回転式3DLiDARで、GLIM\ref{GLIM}を用いて作成した。
GLIMはさまざまな3DLiDARに対応したSLAM手法である。
IMUやオドメトリを入力することでLiDARの値とタイトカップリングで位置推定するため、より正確なSLAMを行うことができる。
3DLiDARの回転式やソリッドステート式だけでなく、深度付きカメラにも対応する。
IMUやオドメトリの有無も選ぶことができるため、利用の敷居はとても低い。

作成したつくばチャレンジの走行範囲全域の地図は図nの通りである。
つくばチャレンジの現場でロボットのセンサを起動し、スタート地点からゴールまで走行した時のセンサデータを入力し作成した。

\subsection{パラメータチューニング}
GLIMのチューニングとしては以下の点を変更した。
\begin{itemize}
    \item k\_correspondences
    \item voxel\_resolution
\end{itemize}

k\_correspondencesは一つのサブマップを構築する際に利用する点群のスキャンの数である。
回転式LiDARの16ラインのものだと点群数が疎であるため、この数を上げないとマッチングに必要な点群数に至らないことがある。
32ラインのLiDARはデフォルトから変更なし、16ラインのLiDARは数を30程度に設定するとうまく地図が構築されることを確認した。

voxel\_resolutionは3次元空間の点群を配置する解像度である。
つくばチャレンジ2024のようなキロメートルオーダーでは、デフォルトの0.1m四方だと、GPUメモリ8GBが飽和した。
今回は0.25m四方に設定したら、研究学園駅前公園を含むすべての範囲で地図を作成することができた。

最後にGLIMのオフラインツールで追加でループクローズを設定する。
SLAMを実施した後、オフラインツールで再度開くと、各サブマップ同士の位置関係をグラフィカルに確認することができる。
このツールで、スタート地点、往路と復路が重なるパイロン地帯、横断歩道の待機場所で再度ループクローズをかけることで、より正確な地図に修正された。

これらの作業を通じて、つくばチャレンジ全域で自己位置推定ができる地図ができた。
3D地図はhttps://google-drive で公開しているので、来年同じ構成で使用したい方は使うと良い。
\section{自己位置推定}
前章で作成した3D地図に対して,ロボットの自己位置を推定をする.
3D地図を利用することで,つくばチャレンジ2024のコース全域でほとんど破綻することなく位置を推定することができた.

自己位置推定手法には,OSSのlidar\_localization\_ros2\cite{Localization}を利用した.

\subsection{lidar\_localization\_ros2}
lidar\_localization\_ros2はROS 2で動作する自己位置推定アルゴリズムである.
3DLiDARの他にIMUやオドメトリのセンサ値を利用してよりロバストに推定することができる.
前章で作成した3D地図をpcdファイルに変換して読み込み,コンフィグで有効にしているセンサを起動すれば,ロボットの自己位置を維持してくれる.

\subsection{パラメータチューニング}
lidar\_localization\_ros2のチューニングとしては以下の点を変更した.
\begin{itemize}
    \item $ndt\_resolution$
    \item $use\_odom$
    \item $use\_imu$
\end{itemize}

$ndt\_resolution$ を小さく設定すると,自己位置が非常に滑らかにプロットされる傾向があった.
一方で,マッチングが失敗する確率が高くなり全く別の位置に吹っ飛んで破綻することが多かった.
初期値が$2.0$であったが,$5.0$にすることで,破綻を防ぐことができた.ただし,プロットが荒く0.1~0.3mほど一瞬ずれることがあった.
ずれることがあっても,すぐに復帰し戻ることと,ロボットの走行速度が遅いことから悪影響なしと判断し,安定性を重視したパラメータに設定した.

$use\_odom$ と $use\_imu$ でホイールオドメトリとIMUの値を自己位置推定に反映できる.

ホイールオドメトリを有効にすると前述の急激な自己位置推定変化を抑制することができる.
モータの回転数は突然極大な値を出力することがなく安定した値を出力するためである.
今回はホイールオドメトリの値を有効に設定した.

IMUはホイールオドメトリのような段差などの不整地によるスリップの影響を受けにくい.
今回は安価なIMUを使用していたため,ある方向のバイアスが強く作用した.
LiDARによるスキャンマッチングと誤マッチングを抑制するホイールオドメトリが働いていたが,
IMUが常に強い横移動を示し続けていて,自己位位置に悪影響を与えていたためIMUの値を無効に設定した.

つくばチャレンジ期間中,走行中に自己位置推定を喪失し,次にどこに向かえば良いかわからない状態が発生することなく安定して動作し続けることができた.
\section{障害物認識}
自己位置とゴールが分かればロボットが進むべき経路を計算できる。
ただし、これだけでは現実にある衝突してはいけない障害物を考慮していない。
LiDARなどのセンサを使って障害物を認識し、それを地図に反映させることで障害物を考慮した経路を計算できる。
多くのロボットでは、2DLiDARを障害物に使用している。
2DLiDARを使うと坂道など地面が写ってしまう。
2DLiDARを使うと机のような物体は脚しか映らない。
3DLiDARを使うことで机の柱と天板の部分の全体の点群を得ることができる。
2DLiDARと異なり、ぶつかりたくない部分を抽出して点群として表現できる。
メカの章に載せた、MID360でを使用した。

\subsection{PointCloudToLaserScan}
これを実現するROSパッケージを作成した。
簡単な手順としては以下の通りだ。
点群をダウンサンプリングする。
LiDAR点群の傾きを設置角度から逆算して水平にする。
法線を各点ごとに計算する。
法線ベクトルから水平なものは地面など問題ないものと判断した。
法線ベクトルから垂直に近いものは壁や障害物と判断した。
計算を単純にするために、法線のZ方向で閾値で区切った。
これで坂道や乗り越え可能な5cm程度の小物は水平と判断し障害物判定できた。
細い鉄パイプやパイロンの根本も障害物として判定できた。
出力した点群をOSSのPointCloudToLaserScanパッケージに入力し、2DLiDARのscanトピックに変換した。
このScanトピックをNavigation2に入力することで、広く使われる2DLiDARのサンプルをそのまま動かすことができた。
つくばチャレンジ環境で使用してパイロンや人、生垣などの大半の障害物を検出することができた。
パイロンは先端の細いところから根本の一番太いところまで観測できた。
この方式では、パイロンの一番太いところをScanとして出力するため、パイロンのぎりぎりを通過することがすくない。
苦手な物体として、背のひくい水平な障害物は検出することが難しい。
平台車。
小さな段ボール。
地面成分が大きく平均化され溶け込んでしまう。

\subsection{性能評価}
2.4GHz CPUシングルスレッドで40msくらい。
8スレッドで分散処理して7msくらい。
PCの電飾消費が深刻だったので、あえてシングルスレッドの実装で走行した。
パッケージはOSSで公開しているので、ぜひ利用してほしい。
\section{ナビゲーション}
\subsection{経路計画}
ROS 2では、経路計画パッケージとしてNavigation2がある。
環境地図と自己位置を入力すれば、指定したゴール座標までの経路を出力してくれる。
計算アルゴリズムの手法が複数用意されている。
各手法がPluginとして実装されており、コンフィグファイルでPlugin名を変更するだけで反映される。
つくばチャレンジ2024では、3DMapを利用して自己位置推定を行なった。
Navigation2にはMapを入力せず、3DMapからの自己位置推定のX,Y,Yawの値をそのまま利用するだけで2Dに転写した。
Navigaiton2には作成したObstacleToScanから2DLiDARを模擬したscanトピックを入力した。
これで2DMapはないが、ロボット周辺の障害物マップは作成される。
これで、ロボットから2〜10m程度の距離にゴールを指定することで短距離の経路を計算してくれる。
この短距離で達成できるゴールを数メートルおきに、完走のゴールまで配置しつづける。
近くのゴールが達成できれば、次のゴールを設定する処理を自動化した。
これがPenguinNav。
penguinnavはお願いします。
SmacPlanner。
経路計画方法はSmacPlannerを利用した。
SmacPlannerはNavigation2のPluginに実装されているため、ConfigファイルでSmacPlannerを選択するだけで利用できる。
SmacPlannerはロボットの形状を考慮した衝突判定をしてくれる。
したがって、自ロボットの投影面積であるポリゴンを指定するだけで、無茶な経路は選ばれなくなる。
また、最小回転半径も指定できるため、なめらかな円弧で走行経路を利用できる。
これにより、障害物ギリギリまで近づいて急に避ける行動を抑えることができた。

\subsection{経路追従}
前節のSmacPlannerで計算された経路をロボットがどのように再現するか計算するアルゴリズムを経路追従アルゴリズムという
Navigation2では、さまざまな経路追従アルゴリズムを利用できる
つくばチャレンジ2024では、DWBをつかった
DWBはロボットが再現可能な速度や加速度の範囲の中で一番良い速度を選択する
最高速度と最低速度と加速度を設定できる
クローラの制御の場合、抵抗が非常に大きく初動に大きなエネルギーを必要とする
最低速度を速く設定し、すぐに停止できるように減速加速度を大きくなるように設定した。
\section{本走行}
つくばチャレンジ2024の本走行では,市庁舎裏の細い通路でロボットを回避し,元の経路に再計画できずに180m地点でリタイアした.
それまでの走行は,自己位置を見失うこともなく,障害物を未検出になることなく動作した.

障害物からは距離を1m程度離れるように設定していたため,狭路で非常にゆっくり進んだ.
そのため,図\ref{fig:honsoukou1}のように後続のロボットが渋滞してしまった.

その後,図\ref{fig:honsoukou2},\ref{fig:honsoukou3}のように,ロボットが追い越し,追い越され,経路上で詰まってしまった.
このロボットは設定したルートに忠実になぞる機能しか搭載していなかったゆえに,
迂回することができず,図\ref{fig:honsoukou4}のように障害物を回避したあと復帰できなくなった.

本年は確認走行区間内でリタイアしたが,
決められた座標の通りに走行する機能までは3DLiDARを使って実現することができた.
\section{結論}
つくばチャレンジ2024では、3DLiDARを利用したナビゲーションシステムを構築することができた。
3DLiDARのナビゲーションシステムは文献が少なく構築することに苦労したが、最低限の機能を揃えることはできた
つくばチャレンジの多くのロボットは3DLiDARを利用しているが、何から手をつければ良いかわからない人はいるはずである
このレポートでは、ハードウェアから自律ナビゲーションシステムの各機能のおおよその仕組みを述べた
来年初参加を考えている人の参考になれば幸いである
環境構築の方法や実際の詳細の使用方法などは引き続きブログで解説していく予定である。
次年度では、決められたルートを走行する上での意思決定機能を追加で実装する
完走するためには、不測の事態でも、認知判断行動することが必須である
完走にむけてナビゲーションシステムをブラッシュアップしていく予定である。

% 参考文献
\bibliographystyle{junsrt}
\bibliography{myrefs}

\end{document}
