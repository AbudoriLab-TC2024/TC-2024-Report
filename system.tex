\section{システム}


\subsection{ハードウェア}
自律走行を行う機体で必要な要素は大きく分けると下記5点となる。

センサ
走行装置
計算機
バッテリー
シャーシ

センサは、自己位置推定や障害物を検出する為に必要となる。
自己位置推定にはLiDARが多く使用されるが、GNSS、カメラといったセンサも使用される。 
走行装置は、ロボットの移動の為に必要となる。
ロボットではモータで車輪を駆動する形式が一般的だが、不整地走行に適したクローラ、段差や階段にも対応できる脚といった機構も使用される。 
計算機は、センサから取得された情報から走行装置への指令を算出する為に必要となる。
大量の計算が必要となる自己位置推定には概ねPCが使用される。
多くのIOを要求されるモータ制御のため、マイコンも使用される。 
バッテリーは、上記の機器を使用する為に必要な電力を供給ために必要となる。
移動ロボットは消費電力低下のため、重量効率の高いLiPoバッテリーが多く使用される。 
シャーシは、その他の機器を接続する為の機構物として必要となる。
状況と要求されるタスクにより、適切にシャーシを作成する必要がある。

\subsubsection{自律走行ロボット:Penguin}
今回、自律走行を実現する基本的な構成を持ちつつ、不整地の走行も可能な機体とすることをコンセプトに、機体を作成した。 
作成した機体とその搭載した機器を、図nに記載する。以降の章では、各要素について詳細を記載していく。

\subsubsection{センサ}
自己位置推定を行うためのLiDARセンサとして、Velodyne社のVLP16を搭載した。
自己位置推定用のLiDARは、遠距離にある環境物(建物、樹木等)をセンシングすることで、自己位置を推定する。
通行人や障害物があった際にレーザー光を阻害し、自己位置推定の精度低下を発生させないために、機体の最上部に搭載した。
また、LiDARは光を使用したセンサであるため、光学窓に傷が発生すると誤検知が発生する。
これを防ぐため、金蔵製のガードを作成した。

ロボットの自律走行を阻害する障害物を検出するLiDAR センサとして、Livox社のMID-360を搭載した。
自律走行用のLiDARは、遠距離をセンシングする必要がないため、機体の下部に設置した。
また、今回の機体は後進を行わないため、障害物の検出範囲は前方向のみとした。
障害物検出は、後述の通り地面の検出を実施するため、障害物検出範囲の地面も検出できる位置とした。
上記をみたす位置として、最終的な搭載位置は図nのような箇所とした。
LiDARを35°前傾させる事で、LiDAR取付位置から250mm先の地面を検出できる形とした。
障害物の検出範囲は、機体進行方向から左右78°となった。
障害物検出用のLiDARも、光学窓への傷を防ぐ為、金属製のガードを上下に設置した。

\subsubsection{走行装置}
走行装置には、CuboRex社のクローラロボット開発プラットフォーム CuGo V4を使用した。
不整地走行に適したクローラを有するプラットフォームを使用する事で、不整地の走破性向上と開発の高速化を図った。

\subsubsection{計算機}
ロボットの制御を行うPCとして、○○社の○○を搭載した。
ノートPCを搭載することにより、ロボットを駆動する電源系列とPCを駆動する計算機の電源分割を実現した。
モータの制御には、CuboRex社製モータドライバを使用した。

\subsubsection{バッテリー}
機体下部に、24V○○Ah のLiFePO4バッテリを搭載した。
本機体は不整地走行の可能性を考慮したため、路面により大きな振動が発生する可能性が考えられた。
想定を超える振動によるバッテリへの悪影響を考え、LiPoバッテリより安全性の高いLiFePO4バッテリーを採用した。
バッテリは機体下部に設置することで、重心の上昇による安定性の低下を防いだ。
バッテリから供給される電源は、ロボットに搭載した電源分配基板により適切に切替・分配・変圧し、PCを除く各機器に接続した。

\subsubsection{シャーシ}
ロボットのシャーシは図nの形とした。
屋外でのデバッグ作業を想定し、日よけの壁を搭載した。
ロボットは図nのような5ユニットで構成し、各部を計10本のボルトで分解できる形とした。
分解箇所には付当てを設置し、複数回の分割。組立を実施しても再現性のあるシャーシとなる形とした。

構造部材にはミスミ社 R形状アルミフレーム を使用することで、衝突しても対象に危害を加えにくい形状を実現した。
突起部・巻き込みが発生しうる箇所には、樹脂製のカバーを搭載した。

ロボットの状態表示のため、アルミフレームの溝部にLEDを搭載した。
これにより、ロボットが自律走行状態か、操作者による操作を実施している状態かを表示した。

\subsection{ソフトウェア}
自律走行システムのソフトウェア

ロボットを走行させるためのソフトウェア構成は以下の通りである。
地図作成。
ロボットの活動範囲内は既知の場所。
現実の場所とリンクした地図をロボットに持たせる。
自己位置推定に使用する。
自己位置推定。
現在の位置がロボットが持っている地図のどこに位置するのか推定する。
現在位置が分かれば、目的までの向かう経路を計算することができる。
障害物認識。
自己位置推定ができても、実際には障害物があってたどり着かないことがほとんどだ。
向かいたい場所までの経路上に障害物がどのようにあるのか反映させる。
経路計画。
ロボットは地図をもっているので目的地点と現在位置があれば2点間の経路を計算できる。
このとき、環境上にある障害物も加味し、障害物に衝突しない経路を計算する。
実際の自律走行では、数メートルおきに達成地点を設置し、細かく経路計画を実施する。
経路追従。
計算した経路にロボットを追従させる。
障害物を回避しながら、柔軟に経路通りにロボットを制御させる必要がある。
次の章から、それぞれの方法で具体的に使用したソフトウェアを紹介する。