\section{はじめに}
\pkg{RBProceedings}文書クラスはW3Cにより策定されている『日本語組版の要件』\cite{JLREQ}に準拠することを目指す\pkg{jlreq}クラスをベースにしている.
ただし,本文書クラスでは紙面スペースの都合上,多くの余白値をかなり詰めるように設定しており,例えば行間は\ruby{外国人参政権}{がい|こく|じん|さん|せい|けん}のようにルビを振れる最小限の余白に設定してある.

論文では,単純なテキストのみならず,しばしば数式
\begin{equation}
P(B\mid A) = \frac{P(A\mid B)P(B)}{P(A)}
\end{equation}
や箇条書き
\begin{itemize}
\item 第1の項目
\item 第2の項目
\end{itemize}
といった構造も用いられるが,これらもよく知られた文書クラス(例えば\pkg{jsarticle}等)と同様のシンタックスで利用できる.