\section{はじめに}
つくばチャレンジは2024年のロボット大賞に選出された。
つくばチャレンジは18年間、課題や取り組みを変えながら日本のロボット技術の進歩に大きく貢献してきた。
さまざまな研究者に大体的に実験できる環境を提供しつづけロボットの技術を成長させてきた。
完走しているチームの技術力は非常に高い。
初めて参加するチームや個人で参加するチームとは技術的な差が大きい。
初参加から完走するまでの道のりは遠い。
その要因は以下の点であると感じた。
完走しているチームの多くは高価な3DLiDARを搭載している。
ROS 2を使用した3Dナビゲーションのノウハウがウェブ上にあまり公開されていない。
つくばチャレンジレポートは昨年のレポートの差分が書かれていることが多い。
つくばチャレンジレポートは特定の課題の解法が書かれていることが多い。
つくばチャレンジレポートは過去のレポートがオープンになっていないため、引用先が見られない。
以上の点から、初学者や新規参入者に対して敷居が高いように感じた。
同時に、完走するために必要なことは以下のように感じた。
屋外の自律走行には初学者ほど3DLiDARがあった方がよい。
複雑なシステムを運用するには、質の良いハードウェアが必要である。
OSSを駆使すればある程度まではロボットを動作させることができる。
OSSを利用するだけでは完走には至らない。
AbudoriLab.チームでは、本走行では確認走行区間をクリアできなかった。
他のロボットが通路を塞いだため新ルートを作れなかった。
決められたルートに何も障害物がない状態であれば、もっと走行することができた。
決められたルートを走行することだけであれば、自己位置を喪失することなく、障害物も発見し走行することができた。
ゼロからつくばチャレンジに参加する人に必要なことがわかるレポートにしたい。
AbudoriLab.が用意したロボットやソフトウェア構成を紹介する。