\section{はじめに}
つくばチャレンジは2024年のロボット大賞に選出\cite{tsukuba}された。
つくばチャレンジは18年間、課題や取り組みを変えながら日本のロボット技術の進歩に大きく貢献してきた。
このつくばチャレンジで完走しているチームの技術力は非常に高い。

この高い技術力を学ぼうと思っても、初学者にとって超えるべきハードルも高い。
あらためて、つくばチャレンジに参加しようとシンポジウムに参加した時のことを思い出してみると、以下の点で難しさを感じた。

屋外環境のナビゲーションは、3DLiDARによるセンシングが効果的であることは、感覚的にわかる。
実際にやってみようと思うと、ROSのチュートリアルや解説記事などでは、
2DLiDARによるナビゲーションを扱われていることが多いが、3DLiDARの記述は多くないことがわかった。
そうなると、現場で実際に手を動かして知見を得る方法しかない。
このつくばチャレンジ2024で3DLiDARを使ったナビゲーションシステムを作ることで、学ぶこととした。

また、現在の日本語のWeb上では、3DLiDARを用いたナビゲーションシステムの知見が少ない。
このつくばチャレンジ2024で構築したノウハウを各分野ごとに公開することで、少しだけでも貢献できると考えた。
必ずしも、正解ではないが、失敗も含めてブログに綴り公開している。
詳しくは\url{https://www.abudorilab.com/entry/2024/07/14/132220}を参照していただきたい。


AbudoriLab.チームでは、以下の観点でつくばチャレンジ2024に参加した。
\begin{itemize}
    \item 3D地図の構築方法を知る
    \item 3D地図による自己位置推定方法を知る
    \item 3DLiDARによる障害物検知を知る
    \item 3DLiDAR使用しながら2Dナビゲーションの方法を知る
\end{itemize}

AbudoriLab.チームの本走行では確認走行区間をクリアできなかった。
しかし、決められたルートを走行することだけであれば、自己位置を喪失することなく、障害物も発見し走行することができた。

このレポートはゼロからつくばチャレンジに参加する人に必要なことがわかるレポートにしたいと考える。
AbudoriLab.が用意したロボットやソフトウェア構成を網羅的に紹介する。