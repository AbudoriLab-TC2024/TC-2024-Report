%#!platex
% NLproceedings サンプル文書.パブリックドメイン.
\documentclass[
  platex, dvipdfmx,  % ワークフローは必ず明示的に指定する
]{rbproceedings}

% パッケージ
% \usepackage{jlreq-deluxe}     % 多書体化(otf パッケージは使用しない)
\usepackage{graphicx,xcolor}  % グラフィックス関連
\usepackage{bxjalipsum}       % ダミーテキスト
\usepackage{pxrubrica}        % ルビ
\usepackage{url}

% 著者用マクロ
\newcommand{\pkg}[1]{\textsf{#1}}
\newcommand{\code}[1]{\texttt{#1}}

\title{\pkg{NLProceedings}文書クラス サンプル文書}
\author{%
  言語太郎 \\ 言語大 \\ \texttt{taro@nlp.example.com}\and
  言語花子 \\ 言語大 \\ \texttt{hanako@nlp.example.com}}

\begin{document}

\maketitle

\section{はじめに}

\pkg{NLProceedings}文書クラスはW3Cにより策定されている『日本語組版の要件』%
\cite{JLREQ}に準拠することを目指す\pkg{jlreq}クラスをベースにしている.
本文書クラスでは紙面スペースの都合上,多くの余白値をかなり詰めるように設定
しており,例えば行間は\ruby{外国人参政権}{がい|こく|じん|さん|せい|けん}の
ようにルビを振るとほとんど余裕がない.

さてNLP分野の論文では,単純なテキストのみならず,しばしば数式
%
\begin{equation}
P(B\mid A) = \frac{P(A\mid B)P(B)}{P(A)}
\end{equation}
%
や箇条書き
%
\begin{itemize}
\item 第一の項目
\item 第二の項目
\end{itemize}
%
といった構造も用いられるが,これらもよく知られた文書クラス(例えば
\pkg{jsarticle})等と同様のシンタックスで利用できる.

\subsection{図表の挿入}

図表についても通常の\LaTeX と同じ方法を用いることができる.

\subsubsection{図について}

図の挿入は通常\pkg{graphicx}パッケージによって行う(図\ref{fig:sample}).
クラスオプションにワークフロー(\code{dvipdfmx}等)を指定していれば,
各パッケージを読み込む際に何度も同じオプションを指定する必要はない.
%
\begin{figure}[h]
\centering
\includegraphics[width=3cm]{example-image-a}
\caption{何らかの図}
\label{fig:sample}
\end{figure}

\subsubsection{表について}

表組みももちろん利用できるが,図とは異なりキャプションは表本体の上に付ける
(表\ref{tab:sample}).
%
\begin{table}[h]
\centering
\caption{適当な表}
\label{tab:sample}
\begin{tabular}{ll}
\hline
日本語 & Japanese \\
英語 & English \\
\hline
\end{tabular}
\end{table}

本文書クラスの仕様については\code{README.md}を参照されたい.以下,いくらか
長さのある文章を組版したときの様子を確認するためのダミーテキストである.

\section{ダミーテキスト}

\jalipsum[1-3]{wagahai}

\jalipsum[4-23]{wagahai}

\begin{thebibliography}{9}
\bibitem{JLREQ}
  W3C日本語組版タスクフォース.日本語組版の要件(日本語版).
  \url{https://www.w3.org/TR/jlreq/},2020.(2020-11-09閲覧)
\end{thebibliography}

\end{document}
