\section{自己位置推定}
前章で作成した3D地図に対して、ロボットの自己位置を表現する。
3D地図で自己位置推定することで、2D地図よりも以下の点で優れていると考えたため採用した。
トラックのような大きな障害物がある場合でも自己位置が破綻しない。
2D地図の場合はバンが止まっている場所に近づいた時、位置が飛んでしまった。
2D地図の場合はロボットの周辺に人だかりがあるとマッチングできない。
2D地図の場合は坂道のように地面が2DLiDARに映る場合、位置が飛んでしまった。
後述のNavigationでは、2Dを利用する。
地面を走行するロボットであるため、2Dの姿勢で十分。
3Dで自己位置推定して、そのうちX,Y,Yawだけを抽出して利用する。
\subsection{Lidar Localization ROS2}
自己位置推定手法としては、OSSのLidarlocalizationros2を利用した。
前章で作成した3D地図をpcdに変換して読み込んだ。
地図はコンフィグファイルに絶対パスを書き込むだけ読み込まれる。
LiDARを起動した状態で自己位置推定アルゴリズムを起動する。
起動した後に初期位置を与える。
コンフィグで初期設定を与えることもできる。
初期姿勢と地図のマッチングが成功すると常にLiDARスキャンマッチをつづけて自己位置を更新し続ける。

\subsection{パラメータチューニング}
設定にチューニングとしては以下の点である。
あああ
いいい
ううう
AAAの点では、あああパラメータが効く
BBBの点では、いいいパラメータが効く
CCCの点では、うううパラメータが効く
XXXという点に注意してパラメータチューニングするとよい。
以下の点で自己位置が破綻することがある。
あああ
いいい
ううう
オドメトリをオンにすることでふっとびを抑えることができた。
IMUはチューニング不足で破綻してしまった。
結果的にXXXに注意すると良い。