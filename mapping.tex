\section{地図作成}
現代の多くの自律走行ロボットは、行動範囲分の環境地図を必要とする。
ロボットが環境地図を持つことにより、既知の環境内で自由に行動することを目指す。

物流倉庫やオフィスビルなどの環境では2Dの地図を使用しても充分に効果を発揮する。
しかし、つくばチャレンジのようなひらけた場所であったり、交通の往来があったりする屋外環境であると2D地図の使い方によっては、性能が不十分になることがある。
このつくばチャレンジ2024では、回転式3DLiDARを利用し3D地図を作成した。
3D地図は複雑な環境や、本走行のように人だかりでロボット周辺に障害物がたくさんあってもロバストに自己位置推定を続けることができた。
回転式3DLiDARは非常に高価であるが、つくばチャレンジでは、無料貸与があるため、参加者は利用できるチャンスがある。
つくばチャレンジ2024では、株式会社アルゴさまより無料で貸与いただいた。
ここにお礼のことば(謝辞行きかどうかは要検討)

\subsection{GLIM}
地図作成には、回転式3DLiDARで、GLIM\cite{GLIM}を用いて作成した。
GLIMはさまざまな3DLiDARに対応したSLAM手法である。
IMUやオドメトリを入力することでLiDARの値とタイトカップリングで位置推定するため、より正確なSLAMを行うことができる。
3DLiDARの回転式やソリッドステート式だけでなく、深度付きカメラにも対応する。
IMUやオドメトリの有無も選ぶことができるため、利用の敷居はとても低い。

作成したつくばチャレンジの走行範囲全域の地図は図nの通りである。
つくばチャレンジの現場でロボットのセンサを起動し、スタート地点からゴールまで走行した時のセンサデータを入力し作成した。

\subsection{パラメータチューニング}
GLIMのチューニングとしては以下の点を変更した。
\begin{itemize}
    \item $k\_correspondences$
    \item $voxel\_resolution$
\end{itemize}

$k\_correspondences$は一つのサブマップを構築する際に利用する点群のスキャンの数である。
回転式LiDARの16ラインのものだと点群が疎であるため、この数を上げないとマッチングに必要な点群数に至らないことがある。
32ラインのLiDARはデフォルトから変更なし、16ラインのLiDARは数を30程度に設定するとうまく地図が構築されることを確認した。

$voxel\_resolution$は3次元空間の点群を配置する解像度である。
つくばチャレンジ2024のようなキロメートルオーダーでは、デフォルトの0.1m四方だと、GPUメモリ8GBが飽和した。
今回は0.25m四方に設定したら、研究学園駅前公園を含むすべての範囲で地図を作成することができた。

最後にGLIMのオフラインツールで追加でループクローズを設定する。
SLAMを実施した後、オフラインツールで再度開くと、各サブマップ同士の位置関係をグラフィカルに確認することができる。
このツールで、スタート地点、往路と復路が重なるパイロン地帯、横断歩道の待機場所で再度ループクローズをかけることで、より正確な地図に修正された。

これらの作業を通じて、つくばチャレンジ全域で自己位置推定ができる地図ができた。
つくばの3D地図はGoogleDriveの\url{https://x.gd/22GO4}で公開しているので、来年同じ構成で使用したい方は使うと良い。