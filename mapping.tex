\section{地図作成}
現代の自律走行の大半は、ロボットの行動範囲の環境地図を作成する。
作成した環境地図をロボットが持つことにより、既知の環境内で自由に行動することを目指す。
ROS 2標準のNavigation2の標準アルゴリズムも含め環境地図は2Dのものが多い。
2D地図は室内など水平であり障害物が単純な環境であれば、とてもよく動作する。
しかし、屋外や広大な環境だと以下のような点でうまくいないことがある。
設置した2DLiDARの設置高さによっては検知できない障害物が多く、無視できないことがある。
上り坂など、本来障害物でないものも障害物として認識することがある。
2D地図で表現しきれない障害物が重要な物体であった場合、精度が大幅に悪化する。
自動車が止まったりするなど環境の変化があると大きく悪影響を受けることがある。
このつくばチャレンジ2024では、3Dの地図を作成し、上記のポイントで複雑な環境でもうまく地図表現をした。
回転式3DLiDARは非常に高価であるが、つくばチャレンジでは、無料貸与がある。
つくばチャレンジ2024では、株式会社アルゴさまより無料で貸与いただいた。
もっていない人でも3DLiDARに挑戦することができる。
ぜひやっていただきたい。
地図作成には、回転式3DLiDARで、GLIM(引用)を用いて作成した。
作成したつくばチャレンジの走行範囲全域の地図は図の通りである。
図
環境構築をした後、全周のルートを走行させた時のロボットの回転式3DLiDARのセンサデータを記録したROSBAGファイルを再生するだけで3D地図が作成できてしまった。
設定にチューニングとしては以下の点である。
あああ
いいい
ううう
AAAの点では、あああパラメータが効く
BBBの点では、いいいパラメータが効く
CCCの点では、うううパラメータが効く
XXXという点に注意してパラメータチューニングするとよい。
最後に、オフライン地図修正ツールが同梱されているのでこれで地図を微修正する。
ループクローズがうまくハマるように指定する。
つくばチャレンジ全域で自己位置推定ができる地図ができた。
3D地図はhttps://google-drive で公開しているので、来年同じ構成で使用したい方は使うと良い。