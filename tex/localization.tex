\section{自己位置推定}
前章で作成した3D地図に対して,ロボットの自己位置を推定をする.
3D地図を利用することで,つくばチャレンジ2024のコース全域でほとんど破綻することなく位置を推定することができた.

自己位置推定手法には,OSSのlidar\_localization\_ros2\cite{Localization}を利用した.

\subsection{lidar\_localization\_ros2}
lidar\_localization\_ros2はROS 2で動作する自己位置推定アルゴリズムである.
3DLiDARの他にIMUやオドメトリのセンサ値を利用してよりロバストに推定することができる.
前章で作成した3D地図をpcdファイルに変換して読み込み,コンフィグで有効にしているセンサを起動すれば,ロボットの自己位置を維持してくれる.

\subsection{パラメータチューニング}
lidar\_localization\_ros2のチューニングとしては以下の点を変更した.
\begin{itemize}
    \item $ndt\_resolution$
    \item $use\_odom$
    \item $use\_imu$
\end{itemize}

$ndt\_resolution$ を小さく設定すると,自己位置が非常に滑らかにプロットされる傾向があった.
一方で,マッチングが失敗する確率が高くなり全く別の位置に吹っ飛んで破綻することが多かった.
初期値が$2.0$であったが,$5.0$にすることで,破綻を防ぐことができた.ただし,プロットが荒く0.1~0.3mほど一瞬ずれることがあった.
ずれることがあっても,すぐに復帰し戻ることと,ロボットの走行速度が遅いことから悪影響なしと判断し,安定性を重視したパラメータに設定した.

$use\_odom$ と $use\_imu$ でホイールオドメトリとIMUの値を自己位置推定に反映できる.

ホイールオドメトリを有効にすると前述の急激な自己位置推定変化を抑制することができる.
モータの回転数は突然極大な値を出力することがなく安定した値を出力するためである.
今回はホイールオドメトリの値を有効に設定した.

IMUはホイールオドメトリのような段差などの不整地によるスリップの影響を受けにくい.
今回は安価なIMUを使用していたため,ある方向のバイアスが強く作用した.
LiDARによるスキャンマッチングと誤マッチングを抑制するホイールオドメトリが働いていたが,
IMUが常に強い横移動を示し続けていて,自己位位置に悪影響を与えていたためIMUの値を無効に設定した.

つくばチャレンジ期間中,走行中に自己位置推定を喪失し,次にどこに向かえば良いかわからない状態が発生することなく安定して動作し続けることができた.