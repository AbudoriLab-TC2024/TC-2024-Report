\section{地図作成}
現代の多くの自律走行ロボットは,行動範囲分の環境地図を必要とする.
ロボットが環境地図を持つことにより,既知の環境内で自由に行動することを目指す.

物流倉庫やオフィスビルなどの環境では2D地図を使用しても充分に効果を発揮する.
しかし,つくばチャレンジのようなひらけた場所であったり,交通の往来があったりする屋外環境であると2D地図の使い方によっては,性能が不十分になることがある.
このつくばチャレンジ2024では,回転式3DLiDARを利用し3D地図を作成した.
3D地図は道路や建物など特徴が異なる環境や,人だかりでロボット周辺に障害物がたくさんあるような環境であってもロバストに自己位置推定を続けることができた.

\subsection{GLIM}
地図作成には,回転式3DLiDARで,GLIM\cite{GLIM}を用いて作成した.
GLIMはさまざまな3DLiDARに対応したSLAM手法である.
IMUやオドメトリの値とLiDARの値を密結合することにより,正確なSLAMを行うことができる.
回転式やソリッドステート式の3DLiDARだけでなく,深度付きカメラにも対応する.
IMUやオドメトリの値の有無も選ぶことができるため,利用の敷居はとても低い.

作成したつくばチャレンジの走行範囲全域の地図は図\ref{fig:tsukuba_map}の通りである.
つくばチャレンジの現場でロボットのセンサを起動し,スタート地点からゴールまで走行した時のセンサデータを入力し地図を作成した.

\subsection{パラメータチューニング}
GLIMのチューニングとしては以下の点を変更した.
\begin{itemize}
    \item $k\_correspondences$
    \item $voxel\_resolution$
\end{itemize}

$k\_correspondences$は一つのサブマップを構築する際に利用する点群のスキャンの数である.
回転式LiDARの16ラインのものだと点群が疎であるため,この数を上げないとマッチングに必要な点群数に至らないことがある.
32ラインのLiDARはデフォルトから変更なし,16ラインのLiDARは数を30程度に設定するとうまく地図が構築されることを確認した.

$voxel\_resolution$は3次元空間の点群を配置する解像度である.
つくばチャレンジ2024のようなキロメートルオーダーでは,デフォルトの0.1m四方だと,GPUメモリ8GBが飽和した.
今回は0.25m四方に設定したら,研究学園駅前公園を含むすべての範囲で地図を作成することができた.

最後にGLIMのオフラインツールでループクローズを設定する.
SLAMを実施した後,オフラインツールで再度開くと,各サブマップ同士の位置関係をグラフィカルに確認することができる.
このツールで,スタート地点,往路と復路が重なるパイロン地帯,横断歩道の待機場所で再度ループクローズをかけることで,より正確な地図に修正された.

これらの作業を通じて,つくばチャレンジ全域で自己位置推定ができる地図ができた.
つくばの3D地図はGoogleDriveの\url{https://x.gd/22GO4}で公開しているので,来年同じ構成で使用したい方は使うと良い.